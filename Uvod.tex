\chapter{Úvod}
Moderné výstavníctvo prechádza zásadnou transformáciou. Návštevníci múzeí a galérií už nehľadajú len pasívne prezeranie exponátov, ale očakávajú príbeh, do ktorého môžu vstúpiť. Tento trend „gamifikácie“ expozícií kladie nové nároky na technické zázemie. Už nestačí len v slučke prehrávať video a svietiť na vitrínu. Systém musí byť schopný reagovať na podnety návštevníka, vetviť príbeh na základe jeho rozhodnutí a synchronizovať tieto zmeny s fyzickými efektmi v reálnom čase – od zmeny osvetlenia, cez pohyb mechanických častí, až po spustenie dymových clon.
Realizácia takýchto interaktívnych scén však v praxi naráža na technologické a finančné bariéry. Komerčné systémy "Show Control" sú robustné, ale pre menšie inštalácie extrémne drahé. Naopak, jednoduché časovače neumožňujú interaktivitu. Častým problémom pri inštaláciách v historických budovách je tiež kabeláž – ťahanie desiatok metrov káblov k motorom a senzorom je často nemožné alebo esteticky neprijateľné.
Cieľom tejto bakalárskej práce je návrh a realizácia modulárneho riadiaceho systému, ktorý tieto problémy rieši kombináciou moderných IoT technológií a vlastnej softvérovej architektúry. Pôvodná myšlienka práce vychádzala z potreby jednoduchého lineárneho spúšťača efektov. Počas analýzy požiadaviek na modernú expozíciu sa však tento koncept transformoval na komplexnejší systém riadený stavovým automatom (State Machine). Tento posun umožňuje nielen lineárne časovanie, ale aj vetvenie deja a reakciu na externé udalosti, čím sa expozícia mení na interaktívny zážitok.
Navrhované riešenie je postavené na bezdrôtovej architektúre klient-server, čo eliminuje potrebu rozsiahlej kabeláže. Centrálnu logiku zabezpečuje minipočítač Raspberry Pi, ktorý komunikuje s distribuovanými aktuátormi postavenými na čipoch ESP32 prostredníctvom protokolu MQTT. Celé správanie systému nie je pevne zakódované vo firmvéri, ale je definované v ľahko upraviteľných konfiguračných súboroch (JSON), čo umožňuje správcom múzea meniť scenár výstavy bez nutnosti programovania. Funkčnosť systému je v závere práce demonštrovaná na prototype „jednej miestnosti“, ktorý integruje ovládanie motorov, svetiel a multimediálneho obsahu v reálnom čase.

