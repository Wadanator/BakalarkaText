\documentclass[a4paper,twoside,12pt]{report}% dvoustranný tisk
%\documentclass[12pt]{report}% jednostranný tisk
% všechny soubory jsou v utf-8
	\usepackage{ucs}% pro kódování UTF-8
	\PrerenderUnicode{} % předkreslení diakritiky, možno přidat/ubrat znaky podle potřeby
							                  

\usepackage[czech]{babel}% čeština
%\usepackage[slovak]{babel}% slovenština
\usepackage[IL2]{fontenc}% csr fonty (pokud jsou nainstalovány česká postscriptová mísma)
%\usepackage[T1]{fontenc}% EC fonty - háčky a čárky jsou k písmenkům připojovány - nehezké


\usepackage[]{diplomka}
\usepackage[]{VSKP} % Sablona dle smernice rektora
%%%
%%%
%%% Vytvoří náležitosti dle směrnice rektora
%%%    Soubor VSKP.tex
%%%
%% Do preambule hlavního souboru vložte následující příkazy:
%%    \usepackage{VSKP}  % Načte styl šablony dle směrnice rektora
%%    %%%
%%%
%%% Vytvoří náležitosti dle směrnice rektora
%%%    Soubor VSKP.tex
%%%
%% Do preambule hlavního souboru vložte následující příkazy:
%%    \usepackage{VSKP}  % Načte styl šablony dle směrnice rektora
%%    %%%
%%%
%%% Vytvoří náležitosti dle směrnice rektora
%%%    Soubor VSKP.tex
%%%
%% Do preambule hlavního souboru vložte následující příkazy:
%%    \usepackage{VSKP}  % Načte styl šablony dle směrnice rektora
%%    \input{VSKP}       % Načte data pro vyplnění šablony
%%    \usepackage{fontspec}  % Pro vkládání OTF fontů (vyžaduje titulní list) - nefunguje v pdfLaTeXu
%%
%% Na začátek hlavního souboru (za \begin{document} ) vložte příkazy pro vysázení desek
%%   \titul% vytiskne titul práce
%%   \abstrakty% vytiskne stránku s abstrakty
%%
%% Použité kódování UTF-8
%%
%%% Generování údajů

\fakulta{Fakulta strojního inženýrství}
\enfakulta{Faculty of Mechanical Engineering}
\adresafakulta{Technická 2896/2, 61669 Brno}

\ustav{Ústav mechaniky těles, mechatroniky a biomechaniky}
\enustav{Institute of Solid Mechanics, Mechatronics and Biomechanics}

% udaje o autorovi

\autor{}{Andrej Žabka}{}  % Jméno autora, 
    % Tituly vložte samostatně, např. \autor{Ing.}{Petra Smékalová}{}
\autorzkr{Žabka, A.}
  % bibliografické jméno

\typstudia{B}
  % M, N, B, D
  % M - Magisterské, N - Navazující magisterské, B - Bakalářské, D-Doktorské
  % U typu studia M a N se liší anglický název

\nazev{Modulární systém pro automatizované řízení muzejních expozic} 
  % Ručně můžete dlouhý text zalomit pomocí " \break "
\ennazev{Modular System for the Automated Control of Museum Exhibits} 
  % Ručně můžete dlouhý text zalomit pomocí " \break "

%vedouci prace
\vedouci{Ing.}{Michal Bastl}{, Ph.D.}
\citacevedouci{, Vedoucí bakalářské
práce: Ing. Michal Bastl, Ph.D..} % Označení vedoucího práce pro citaci záv. práce. Musí být ukončeno tečkou.

\datumobhajoby{neuvedeno}
\abstrakt{\noindent } 
  % Před "\n" vložit další "\n"
\enabstrakt{\noindent } 
  % Před "\n" vložit další "\n"
\klicovaslova{\noindent } % Před "\n" vložit další "\n"
\enklicovaslova{\noindent } % Před "\n" vložit další "\n"
  
%%
%%   Konec generování údajů
%%


%%
%%   Vlastní vysázení desek umístnit na začátek práce
%%
%\titul% vytiskne titul práce
%\abstrakty% vytiskne stránku s abstrakty
       % Načte data pro vyplnění šablony
%%    \usepackage{fontspec}  % Pro vkládání OTF fontů (vyžaduje titulní list) - nefunguje v pdfLaTeXu
%%
%% Na začátek hlavního souboru (za \begin{document} ) vložte příkazy pro vysázení desek
%%   \titul% vytiskne titul práce
%%   \abstrakty% vytiskne stránku s abstrakty
%%
%% Použité kódování UTF-8
%%
%%% Generování údajů

\fakulta{Fakulta strojního inženýrství}
\enfakulta{Faculty of Mechanical Engineering}
\adresafakulta{Technická 2896/2, 61669 Brno}

\ustav{Ústav mechaniky těles, mechatroniky a biomechaniky}
\enustav{Institute of Solid Mechanics, Mechatronics and Biomechanics}

% udaje o autorovi

\autor{}{Andrej Žabka}{}  % Jméno autora, 
    % Tituly vložte samostatně, např. \autor{Ing.}{Petra Smékalová}{}
\autorzkr{Žabka, A.}
  % bibliografické jméno

\typstudia{B}
  % M, N, B, D
  % M - Magisterské, N - Navazující magisterské, B - Bakalářské, D-Doktorské
  % U typu studia M a N se liší anglický název

\nazev{Modulární systém pro automatizované řízení muzejních expozic} 
  % Ručně můžete dlouhý text zalomit pomocí " \break "
\ennazev{Modular System for the Automated Control of Museum Exhibits} 
  % Ručně můžete dlouhý text zalomit pomocí " \break "

%vedouci prace
\vedouci{Ing.}{Michal Bastl}{, Ph.D.}
\citacevedouci{, Vedoucí bakalářské
práce: Ing. Michal Bastl, Ph.D..} % Označení vedoucího práce pro citaci záv. práce. Musí být ukončeno tečkou.

\datumobhajoby{neuvedeno}
\abstrakt{\noindent } 
  % Před "\n" vložit další "\n"
\enabstrakt{\noindent } 
  % Před "\n" vložit další "\n"
\klicovaslova{\noindent } % Před "\n" vložit další "\n"
\enklicovaslova{\noindent } % Před "\n" vložit další "\n"
  
%%
%%   Konec generování údajů
%%


%%
%%   Vlastní vysázení desek umístnit na začátek práce
%%
%\titul% vytiskne titul práce
%\abstrakty% vytiskne stránku s abstrakty
       % Načte data pro vyplnění šablony
%%    \usepackage{fontspec}  % Pro vkládání OTF fontů (vyžaduje titulní list) - nefunguje v pdfLaTeXu
%%
%% Na začátek hlavního souboru (za \begin{document} ) vložte příkazy pro vysázení desek
%%   \titul% vytiskne titul práce
%%   \abstrakty% vytiskne stránku s abstrakty
%%
%% Použité kódování UTF-8
%%
%%% Generování údajů

\fakulta{Fakulta strojního inženýrství}
\enfakulta{Faculty of Mechanical Engineering}
\adresafakulta{Technická 2896/2, 61669 Brno}

\ustav{Ústav mechaniky těles, mechatroniky a biomechaniky}
\enustav{Institute of Solid Mechanics, Mechatronics and Biomechanics}

% udaje o autorovi

\autor{}{Andrej Žabka}{}  % Jméno autora, 
    % Tituly vložte samostatně, např. \autor{Ing.}{Petra Smékalová}{}
\autorzkr{Žabka, A.}
  % bibliografické jméno

\typstudia{B}
  % M, N, B, D
  % M - Magisterské, N - Navazující magisterské, B - Bakalářské, D-Doktorské
  % U typu studia M a N se liší anglický název

\nazev{Modulární systém pro automatizované řízení muzejních expozic} 
  % Ručně můžete dlouhý text zalomit pomocí " \break "
\ennazev{Modular System for the Automated Control of Museum Exhibits} 
  % Ručně můžete dlouhý text zalomit pomocí " \break "

%vedouci prace
\vedouci{Ing.}{Michal Bastl}{, Ph.D.}
\citacevedouci{, Vedoucí bakalářské
práce: Ing. Michal Bastl, Ph.D..} % Označení vedoucího práce pro citaci záv. práce. Musí být ukončeno tečkou.

\datumobhajoby{neuvedeno}
\abstrakt{\noindent } 
  % Před "\n" vložit další "\n"
\enabstrakt{\noindent } 
  % Před "\n" vložit další "\n"
\klicovaslova{\noindent } % Před "\n" vložit další "\n"
\enklicovaslova{\noindent } % Před "\n" vložit další "\n"
  
%%
%%   Konec generování údajů
%%


%%
%%   Vlastní vysázení desek umístnit na začátek práce
%%
%\titul% vytiskne titul práce
%\abstrakty% vytiskne stránku s abstrakty
 % Uvodni desky atd dle smernice rektora
\splithyphens% při rozdělování slov se spojovníkem opakuj spojovník
\usepackage[pdftitle={\typpracetxt},
            pdfauthor={\autortxt},
            bookmarks=true,
            pdfencoding=unicode,
            linkcolor=blue,
            colorlinks=true,
            breaklinks=true]{hyperref}
%\usepackage[pdftex]{graphicx}
% Pro vytvoření titulního listu je potreba další balíček
\usepackage{fontspec}  % Pro vkládání OTF fontů (vyžaduje titulní list) - nefunguje v pdfLaTeXu
% Pro vložení titulního listu staženého ze Studisu stačí jen vkládáni PDF
\usepackage{pdfpages} % Pro vkladání PDF souborů (s titulním listem apod.)
\DeclareGraphicsExtensions{.png,.pdf}




\begin{document}

%% Vložení titulního listu staženého ze Studisu, soubor vložte do složky pdf
%\includepdf[pages=1,offset=15.4mm -1in]%
%  {pdf/titulnilist.pdf}% název souboru nesmí obsahovat mezery!
% Alternativně vysázet titulní list LuaLaTeXem
\titul% vytiskne titul práce

\abstrakty% vytiskne stránku s abstrakty


\prohlaseni{Týmto čestne vyhlasujem, že som bakalársku prácu vypracovala samostatne na základe
citovanej literatúry a konzultácií s vedúcim práce, a že som v zozname literatúry uviedla
všetky použité informačné zdroje}% prohlášení,
\podekovani{Poďakovanie tu pridaj}% poděkování, nepovinné

% vlastní práce
\obsah% vytiskne obsah

%
%  vlastni text
%
\chapter{Úvod}
Moderné výstavníctvo prechádza zásadnou transformáciou. Návštevníci múzeí a galérií už nehľadajú len pasívne prezeranie exponátov, ale očakávajú príbeh, do ktorého môžu vstúpiť. Tento trend „gamifikácie“ expozícií kladie nové nároky na technické zázemie. Už nestačí len v slučke prehrávať video a svietiť na vitrínu. Systém musí byť schopný reagovať na podnety návštevníka, vetviť príbeh na základe jeho rozhodnutí a synchronizovať tieto zmeny s fyzickými efektmi v reálnom čase – od zmeny osvetlenia, cez pohyb mechanických častí, až po spustenie dymových clon.
Realizácia takýchto interaktívnych scén však v praxi naráža na technologické a finančné bariéry. Komerčné systémy "Show Control" sú robustné, ale pre menšie inštalácie extrémne drahé. Naopak, jednoduché časovače neumožňujú interaktivitu. Častým problémom pri inštaláciách v historických budovách je tiež kabeláž – ťahanie desiatok metrov káblov k motorom a senzorom je často nemožné alebo esteticky neprijateľné.
Cieľom tejto bakalárskej práce je návrh a realizácia modulárneho riadiaceho systému, ktorý tieto problémy rieši kombináciou moderných IoT technológií a vlastnej softvérovej architektúry. Pôvodná myšlienka práce vychádzala z potreby jednoduchého lineárneho spúšťača efektov. Počas analýzy požiadaviek na modernú expozíciu sa však tento koncept transformoval na komplexnejší systém riadený stavovým automatom (State Machine). Tento posun umožňuje nielen lineárne časovanie, ale aj vetvenie deja a reakciu na externé udalosti, čím sa expozícia mení na interaktívny zážitok.
Navrhované riešenie je postavené na bezdrôtovej architektúre klient-server, čo eliminuje potrebu rozsiahlej kabeláže. Centrálnu logiku zabezpečuje minipočítač Raspberry Pi, ktorý komunikuje s distribuovanými aktuátormi postavenými na čipoch ESP32 prostredníctvom protokolu MQTT. Celé správanie systému nie je pevne zakódované vo firmvéri, ale je definované v ľahko upraviteľných konfiguračných súboroch (JSON), čo umožňuje správcom múzea meniť scenár výstavy bez nutnosti programovania. Funkčnosť systému je v závere práce demonštrovaná na prototype „jednej miestnosti“, ktorý integruje ovládanie motorov, svetiel a multimediálneho obsahu v reálnom čase.

% nutné
%
% sem vlastni opsany text, možno vložit více souborů (nejlépe pro každou kapitolu zvláštní soubor)
\chapter{Súčasný stav problematiky}
\label{SucasnyStav}

Cieľom tejto kapitoly je analyzovať súčasný stav technických prostriedkov používaných v~modernom výstavníctve, multimediálnych inštaláciách a~automatizácii. Kapitola poskytuje kritický prehľad dostupných hardvérových a~softvérových platforiem s~dôrazom na ich vhodnosť pre realizáciu interaktívnych expozícií. Zároveň definuje kľúčové technické požiadavky na komunikačné protokoly v~distribuovaných systémoch.

\section{Existujúce prístupy k~riadeniu expozícií}
\label{existujuce-pristupy}

Pri návrhu techniky pre interaktívne expozície dnes neexistuje len jedno správne riešenie. V~praxi sa však najčastejšie využívajú tri hlavné spôsoby. Realizátori si zvyčajne vyberajú medzi priemyselnými automatmi (PLC), IoT platformami s~vizuálnym programovaním alebo profesionálnymi komerčnými systémami. Táto kapitola popisuje tieto tri kategórie a~hodnotí, či sú vhodné pre naše potreby -- teda pre menšie a~cenovo dostupné múzejné inštalácie.

\subsection{Priemyselné programovateľné automaty (PLC)}
\label{plc}

V~oblasti všeobecnej automatizácie sú dlhoročným štandardom programovateľné logické automaty (PLC -- Programmable Logic Controllers). Tieto systémy, pôvodne vyvinuté pre potreby výrobného priemyslu, sa vyznačujú extrémnou spoľahlivosťou, robustnosťou a~vysokou odolnosťou voči elektromagnetickému rušeniu, prachu a~vibráciám, čo ich predurčuje na nasadenie v~náročných podmienkach [ZDROJ]. Programovanie prebieha zvyčajne v~štandardizovaných jazykoch podľa normy IEC 61131-3 (napr. Ladder Diagram alebo Structured Text).

\textbf{Kritické zhodnotenie:} Napriek vysokej spoľahlivosti naráža nasadenie PLC v~kontexte interaktívnych multimediálnych expozícií na zásadné limity, ktoré robia toto riešenie neefektívnym pre menšie a~stredné projekty:

\begin{enumerate}
\item \textbf{Absencia natívnej podpory multimédií:} PLC sú navrhnuté na spracovanie logických signálov, nedisponujú však rozhraniami pre audiovizuálny výstup (HDMI, Audio Jack). Pre realizáciu požiadaviek, ako je prehrávanie videa, spúšťanie zvukových stôp či zobrazovanie grafického rozhrania, je nevyhnutné PLC doplniť o~externé priemyselné prehrávače (napr. BrightSign) alebo počítače. PLC v~takomto zapojení funguje len ako spúšťač (trigger), čo zvyšuje hardvérovú zložitosť systému.

\item \textbf{Komplexná synchronizácia a~spätná väzba:} Realizácia interaktívnych scenárov, kde systém musí čakať na dokončenie audio stopy pred spustením ďalšej akcie (napr. výber cesty v~príbehu), je pri PLC komplikovaná. Keďže PLC \uv{nevidí} stav prehrávania média, je nutné implementovať obojsmernú komunikáciu (handshaking) cez sériové rozhrania (RS-232) alebo sieťové protokoly (Modbus TCP, UDP). To vyžaduje pokročilé programovanie na strane PLC aj mediálneho servera, pričom akákoľvek zmena v~dĺžke videa vyžaduje preprogramovanie časovačov v~logike automatu.

\item \textbf{Ekonomická a~inštalačná náročnosť:} Okrem vysokej ceny samotných centrálnych jednotiek (CPU) a~licencií, PLC typicky využívajú centralizovanú topológiu typu \uv{hviezda}. To vyžaduje vedenie individuálnej kabeláže od každého senzora a~aktuátora až do centrálneho rozvádzača [ZDROJ]. V~pamiatkovo chránených objektoch alebo pri dynamických výstavách je tento prístup často nerealizovateľný z~estetických a~stavebných dôvodov.
\end{enumerate}

\subsection{Vizuálne IoT platformy (Node-RED)}
\label{node-red}

S~nástupom internetu vecí (IoT) sa v~komunite tvorcov a~integrátorov rozšírili nástroje pre vizuálne programovanie, pričom najvýraznejším zástupcom je platforma Node-RED (pôvodne vyvinutá spoločnosťou IBM). Tento nástroj, postavený na technológii Node.js, umožňuje konfigurovať logiku aplikácie pomocou grafického prepájania funkčných blokov (nodes) a~tokov dát (flows) v~prehľadnom webovom rozhraní [ZDROJ]. Vďaka širokej podpore protokolov (vrátane MQTT) a~hardvérových rozhraní (GPIO) je Node-RED často prvou voľbou pre prototypovanie.

\textbf{Kritické zhodnotenie:} Hoci je Node-RED vysoko efektívny pre jednoduché úlohy typu \uv{akcia -- reakcia} (Event-Driven) v~domácej automatizácii, pri návrhu komplexného riadiaceho systému pre múzeá naráža na limity architektúry a~udržiavateľnosti:

\begin{enumerate}
\item \textbf{Problém pri zložitých scenároch:} Vizuálne programovanie je intuitívne pri lineárnych dejoch. Avšak implementácia komplexného stavového automatu (State Machine), ktorý obsahuje vetvenie deja, cykly, prerušenia (napr. pauza videa) a~spracovanie chýb, vedie v~Node-RED k~vytvoreniu neprehľadnej siete prepojení. Grafická reprezentácia takejto logiky sa stáva vizuálne chaotickou a~ťažko laditeľnou v~porovnaní so štruktúrovaným kódom vo vyššom programovacom jazyku (Python/C++).

\item \textbf{Obmedzená práca s~multimédiami:} Node-RED funguje primárne ako backendová služba a~nemá natívne nástroje na priame renderovanie videa alebo audia na grafický výstup zariadenia (HDMI). Prehrávanie multimédií je nutné riešiť nepriamo -- volaním systémových procesov (napr. exec príkaz pre omxplayer) cez príkazový riadok. Tento prístup sťažuje synchronizáciu a~získavanie spätnej väzby o~stave prehrávania (napr. detekcia konca videa), čo je pre plynulý chod interaktívnej expozície kľúčové.

\item \textbf{Náročná správa konfigurácií:} Logika v~Node-RED je pevne zviazaná s~vizuálnym rozložením toku. Na rozdiel od navrhovaného riešenia, kde je scenár výstavy oddelený do jednoduchého externého konfiguračného súboru (JSON), v~Node-RED by zmena parametrov (napr. dĺžka trvania efektu) vyžadovala zásah priamo do štruktúry programu v~editore. To znemožňuje, aby si scenár upravoval netechnický personál múzea bez rizika poškodenia funkčnosti systému.
\end{enumerate}

\subsection{Komerčné \uv{Show Control} systémy}
\label{show-control}

Na vrchole pyramídy existujúcich riešení stoja systémy ako Crestron, AMX alebo Medialon. Tieto platformy predstavujú profesionálny štandard priamo určený na riadenie komplexných audiovizuálnych inštalácií, divadelnej techniky a~inteligentných budov [ZDROJ]. Ponúkajú dedikovaný hardvér aj softvér optimalizovaný pre multimédiá.

\textbf{Kritické zhodnotenie:} Napriek ich technickej vyspelosti sú tieto systémy spravidla založené na uzavretých (proprietary) technológiách. Sú finančne extrémne náročné nielen z~hľadiska hardvéru, ale aj licencií vývojových prostredí a~následného servisu. Uzatvorený ekosystém často neumožňuje jednoduchú integráciu vlastného hardvéru (napr. mikrokontrolérov tretích strán) a~uzamyká používateľa do produktového radu jedného výrobcu. Tento fakt je v~priamom rozpore s~požiadavkou zadania tejto práce na vytvorenie otvoreného, modulárneho a~cenovo dostupného riešenia.

\section{Komunikačné protokoly a~dátové formáty}
\label{komunikacne-protokoly}

Pre zabezpečenie robustnej, modulárnej a~nízko-latenčnej výmeny dát v~distribuovanom systéme je kľúčová voľba vhodného komunikačného štandardu. Keďže navrhovaná architektúra systému predpokladá komunikáciu centrálnej riadiacej jednotky s~množstvom bezdrôtových periférií v~reálnom čase, výber protokolov musel zohľadňovať obmedzený výpočtový výkon mikrokontrolérov a~požiadavku na minimalizáciu réžie prenosu (overhead).

\subsection{Protokol MQTT}
\label{mqtt}

Ako chrbticový komunikačný štandard bol zvolený protokol MQTT (Message Queuing Telemetry Transport). Ide o~odľahčený sieťový protokol pracujúci nad vrstvou TCP/IP, ktorý sa stal de facto štandardom v~oblasti Internetu vecí (IoT).

Na rozdiel od tradičného modelu klient-server (napr. HTTP), kde musí klient aktívne a~opakovane dopytovať server o~zmenu stavu (Polling), je MQTT založené na architektúre Publish/Subscribe. Tento model využíva prostredníka -- tzv. Broker. Broker prijíma správy od vysielačov (Publishers) a~okamžite ich distribuuje všetkým klientom, ktorí sú prihlásení na odber danej témy (Subscribers). Táto architektúra zabezpečuje úplné oddelenie (decoupling) odosielateľa od príjemcu, čo je pre modulárny systém kľúčové -- centrálna jednotka nemusí poznať IP adresy jednotlivých svetiel, stačí, ak posiela príkazy do správnej témy.

Pre potreby riadenia expozície boli využité špecifické vlastnosti protokolu:

\begin{itemize}
\item \textbf{Hierarchia tém (Topics):} Protokol umožňuje organizovať komunikáciu do stromovej štruktúry pomocou lomítok. Pre tento projekt bola navrhnutá schéma miestnosť/typ\_zariadenia/ID/príkaz (napr. room1/lights/spotlight\_main/set). To umožňuje nielen precízne adresovanie jednotlivých uzlov, ale pomocou divokých kariet (wildcards) aj hromadné ovládanie celých skupín zariadení jedným príkazom.

\item \textbf{Minimalizácia dátového toku:} Hlavička MQTT paketu má veľkosť len 2 bajty. Táto extrémna efektivita znižuje zaťaženie lokálnej Wi-Fi siete a~umožňuje rýchle spracovanie správ aj na jednoduchých 8-bitových alebo 32-bitových mikrokontroléroch.

\item \textbf{Kvalita služby (QoS -- Quality of Service):} Protokol definuje tri úrovne potvrdenia doručenia. Pre multimediálne systémy reálneho času sa paradoxne javí ako najvhodnejšia najnižšia úroveň QoS 0 (At most once). V~aplikáciách vyžadujúcich synchronizáciu (napr. blikanie svetla do hudby) je výhodnejšie prípadný oneskorený paket zahodiť a~spracovať až nasledujúci príkaz v~poradí, než blokovať celú komunikačnú zbernicu čakaním na potvrdenie doručenia (ACK) a~opakovaným odosielaním, čo by viedlo k~viditeľnej desynchronizácii (lagu) [ZDROJ].
\end{itemize}

\subsection{Dátové formáty pre definíciu scén}
\label{datove-formaty}

Pre štruktúrovanie prenášaných dát (Payload) a~ukladanie konfigurácie scenárov bolo nutné zvoliť vhodný serializačný formát. Z~technického hľadiska je na zvolenom hardvéri (Raspberry Pi a~ESP32) možné implementovať podporu pre ktorýkoľvek z~bežných textových štandardov: XML, YAML aj JSON. Výber preto závisel od porovnania ich efektivity, odolnosti voči chybám a~dostupnosti optimalizovaných knižníc.

\subsubsection{XML (eXtensible Markup Language)}
\label{xml}

XML bol dlhé roky priemyselným štandardom a~jeho spracovanie je na ESP32 možné pomocou existujúcich parserov.

\textbf{Nevýhoda:} Pre IoT aplikácie bol vyhodnotený ako neefektívny kvôli svojej \uv{verbosnosti} (veľkému objemu dát potrebných na zápis jednoduchej informácie). To zbytočne zaťažuje sieť a~pamäť mikrokontroléra, keďže parsovanie XML je výpočtovo náročnejšie než u~modernejších formátov.

\subsubsection{YAML (YAML Ain't Markup Language)}
\label{yaml}

Tento formát je vizuálne najčistejší a~je často využívaný v~konfiguráciách serverov. Jeho implementácia na ESP32 je realizovateľná, no prináša prevádzkové riziká.

\textbf{Nevýhoda:} Jeho štruktúra je striktne definovaná odsadením (indentation). To predstavuje kritické riziko pri akejkoľvek manipulácii so súborom -- jediná chýbajúca medzera alebo zámena medzerníka za tabulátor môže znefunkčniť celý konfiguračný súbor. Navyše, knižnice pre parsovanie YAML na platforme ESP32 nie sú tak vysoko optimalizované a~pamäťovo úsporné ako v~prípade formátu JSON.

\subsubsection{JSON (JavaScript Object Notation) -- Zvolené riešenie}
\label{json}

Formát JSON bol vybraný ako najvhodnejší kompromis medzi čitateľnosťou a~technickou efektivitou. Je natívne podporovaný väčšinou moderných jazykov a~na ohraničenie dát používa zátvorky \{\}.

Hlavné dôvody pre voľbu JSON:

\begin{itemize}
\item \textbf{Robustnosť:} Na rozdiel od YAML, JSON ignoruje biele znaky (medzery, nové riadky), čo eliminuje chyby spôsobené zlým formátovaním textu.

\item \textbf{Univerzálnosť:} JSON je prirodzeným jazykom pre webové technológie aj Python, čo zjednodušuje prepojenie všetkých vrstiev systému (Backend -- Frontend -- Firmvér) bez nutnosti konverzie dát.
\end{itemize}

\section{Hardvérové platformy}
\label{hardverove-platformy}

Realizácia distribuovaného riadiaceho systému vyžaduje rozdelenie hardvérovej architektúry na dve úrovne: centrálnu výpočtovú jednotku (Backend) a~koncové výkonové uzly (Periférie).

\subsection{Centrálna riadiaca jednotka (Raspberry Pi)}
\label{raspberry-pi}

Pre serverovú časť systému, ktorá zabezpečuje aplikačnú logiku, sieťovú komunikáciu (MQTT Broker) a~prehrávanie multimédií, sú vhodné jednodoskové počítače (SBC -- Single Board Computer). Ako referenčná platforma sa v~tejto kategórii dlhodobo uvádza Raspberry Pi (konkrétne modely 4 a~5).

Vďaka plnohodnotnému operačnému systému na báze Linuxu (Raspberry Pi OS) poskytuje táto platforma dostatočný výkon pre beh viacerých paralelných služieb, interpretáciu skriptovacích jazykov (Python) a~hardvérovo akcelerované prehrávanie videa cez HDMI rozhranie [ZDROJ]. V~porovnaní s~klasickými osobnými počítačmi (x86 PC) ponúka Raspberry Pi výrazne nižšiu spotrebu energie a~kompaktnejšie rozmery, čo uľahčuje fyzickú integráciu priamo do nábytku expozície alebo do malých osobitných krabičiek.

\subsection{Mikrokontroléry pre koncové uzly}
\label{mikrokontrolery}

Pre realizáciu bezdrôtových koncových bodov, ktoré priamo ovládajú hardvér (motory, svetlá, senzory), je nutné zvoliť vhodný mikrokontrolér. V~oblasti hobby a~poloprofesionálneho IoT sú najčastejšie porovnávané platformy Arduino a~Espressif ESP32.

\begin{itemize}
\item \textbf{Platforma Arduino (Uno/Nano):} Tieto dosky sú založené na staršej architektúre 8-bitových procesorov (AVR) s~nízkou taktovacou frekvenciou (zvyčajne 16\,MHz) a~veľmi malou pamäťou RAM (rádovo v~kilobajtoch). Ich hlavným nedostatkom pre moderné IoT aplikácie je absencia integrovanej sieťovej konektivity. Použitie externých prídavných modulov (Ethernet alebo Wi-Fi shield) komplikuje hardvérový návrh, zvyšuje cenu a~zavádza ďalší bod možnej poruchy [ZDROJ].

\item \textbf{Platforma ESP32:} Ide o~moderný 32-bitový mikrokontrolér (architektúra Xtensa alebo RISC-V) s~frekvenciou až 240\,MHz. Jeho kľúčovou výhodou je plná integrácia Wi-Fi a~Bluetooth rozhrania priamo na čipe (SoC). Disponuje výrazne vyšším výpočtovým výkonom a~väčšou pamäťou SRAM, čo je nevyhnutné pre spracovanie textových protokolov (JSON parsovanie) a~pre zabezpečenú šifrovanú komunikáciu [ZDROJ]. Vynikajúci pomer ceny a~výkonu robí z~ESP32 ideálneho kandidáta pre inteligentné distribuované uzly navrhovaného systému.
\end{itemize}

\subsection{Výkonové rozhrania (Aktuátory)}
\label{aktuatory}

\subsubsection{Ovládanie jednosmerných motorov}
\label{ovladanie-motorov}

Pre plynulé ovládanie smeru a~rýchlosti jednosmerných motorov sa v~robotike a~automatizácii využíva topológia H-mostíka. Tento obvod umožňuje zmenu polarity napätia na svorkách motora, čím sa mení smer otáčania. Rýchlosť sa následne reguluje pomocou pulzne šírkovej modulácie (PWM). V~praxi sa bežne používajú dve kategórie budičov:

\begin{itemize}
\item \textbf{Monolitické integrované obvody:} (Napríklad staršie typy rodiny L298). Sú jednoduché na použitie, ale vyznačujú sa vysokým úbytkom napätia na vnútorných tranzistoroch, čo vedie k~výraznému zahrievaniu a~potrebe chladenia aj pri menších výkonoch.

\item \textbf{Moderné polomostíky (Half-Bridge Drivers):} Využívajú technológiu MOSFET s~veľmi nízkym vnútorným odporom. Tieto obvody (často používané v~automotive priemysle) zvládajú rádovo vyššie prúdy s~minimálnymi tepelnými stratami a~často obsahujú integrované ochrany proti skratu a~prehriatiu.
\end{itemize}

\subsubsection{Spínanie sieťových záťaží (230\,V)}
\label{spinanie-230v}

Pre ovládanie hlavného osvetlenia sa využívajú relé. Okrem klasických elektromechanických relé sa v~múzejníctve čoraz častejšie uplatňujú polovodičové relé (SSR -- Solid State Relay). Ich zásadnou výhodou oproti mechanickým kontaktom je absencia pohyblivých častí, čo zaručuje absolútne tichú prevádzku a~výrazne dlhšiu životnosť pri častom spínaní efektov.

\section{Zhrnutie rešerše}
\label{zhrnutie-reserse}

Analýza dostupných technických riešení preukázala, že pre špecifické potreby modulárnej, flexibilnej a~cenovo dostupnej muzeálnej inštalácie nie sú optimálne ani robustné priemyselné PLC (kvôli vysokej cene a~nutnosti centralizovanej kabeláže), ani uzavreté komerčné systémy \uv{Show Control}.
\chapter{Návrh a~implementácia riadiaceho systému}
\label{NavrhImplementacia}

Cieľom praktickej časti bakalárskej práce bol návrh a~následná realizácia modulárnej architektúry, ktorá integruje závery z~teoretickej rešerše do funkčného prototypu. Kapitola je rozdelená do dvoch hlavných celkov: analytickej časti, ktorá definuje požiadavky a~architektúru systému, a~implementačnej časti, ktorá popisuje konkrétnu realizáciu hardvérových a~softvérových komponentov.

\section{Špecifikácia požiadaviek}
\label{specifikacia-poziadaviek}

Prvým krokom pri návrhu systému bola definícia kľúčových vlastností, ktoré musí výsledné riešenie spĺňať, aby bolo použiteľné v~reálnej muzeálnej praxi.

\subsection{Funkčné požiadavky}
\label{funkcne-poziadavky}

Funkčné požiadavky definujú konkrétne správanie systému z~pohľadu jeho každodennej prevádzky a~interakcie s~personálom múzea:

\begin{itemize}
\item \textbf{Automatizovaná orchestrácia scén:} Systém musí byť schopný na základe časovej osi (timeline) v~reálnom čase synchronizovať prehrávanie multimediálneho obsahu (video/audio) s~fyzickými efektmi (svetlá, motory).

\item \textbf{Interaktivita a~vetvenie deja:} Systém musí disponovať logikou pre spracovanie externých vstupov (tlačidlá, senzory pohybu) a~na ich základe dynamicky meniť priebeh expozície (napr. prepnutie na inú vetvu príbehu).

\item \textbf{Prevádzkový monitoring (Dashboard):} Pre potreby netechnickej obsluhy musí systém poskytovať webové rozhranie dostupné v~lokálnej sieti. Toto rozhranie musí zobrazovať aktuálny stav systému (napr. \uv{Prehráva sa scéna XY}, \uv{Nečinný}) a~poskytovať spätnú väzbu o~pripojení periférií.

\item \textbf{Manuálne riadenie a~bezpečnosť:} Systém musí umožňovať operátorovi manuálne vyvolať konkrétnu scénu a, čo je kritické, musí obsahovať funkciu núdzového zastavenia (Emergency Stop). Táto funkcia musí okamžite prerušiť všetky procesy, zastaviť motory a~uviesť miestnosť do bezpečného stavu.
\end{itemize}

\subsection{Nefunkčné požiadavky}
\label{nefunkcne-poziadavky}

Nefunkčné požiadavky definujú kvalitatívne atribúty systému:

\begin{itemize}
\item \textbf{Stabilita a~dostupnosť:} Keďže expozície bežia často v~režime 24/7 bez prítomnosti technika, systém musí byť odolný voči výpadkom a~disponovať mechanizmom automatického reštartu (Watchdog).

\item \textbf{Nízka latencia:} Odozva systému na interakciu návštevníka musí byť okamžitá (pod 100\,ms), aby bol zážitok plynulý.

\item \textbf{Modularita:} Architektúra musí umožňovať jednoduché pridávanie nových zariadení (škálovateľnosť) bez nutnosti meniť jadro systému.

\item \textbf{Cena:} Riešenie musí byť finančne dostupné pre menšie múzeá, postavené na bežne dostupnom hardvéri (COTS -- Commercial Off-The-Shelf).
\end{itemize}

\section{Návrh architektúry systému}
\label{navrh-architektury}

Na základe stanovených požiadaviek bola navrhnutá architektúra, ktorá je rozdelená do troch logických vrstiev: hardvérovej, komunikačnej a~softvérovej.

\subsection{Topológia siete (Decentralizovaný model)}
\label{topologia-siete}

Architektúra systému bola navrhnutá ako plne decentralizovaná. Na rozdiel od tradičných systémov, kde jeden centrálny server riadi celú budovu, je v~tomto návrhu základnou stavebnou jednotkou autonómny riadič pridelený každej expozičnej miestnosti osobitne.

Tento prístup, definovaný ako \uv{jedna miestnosť = jeden systém}, prináša kľúčové prevádzkové výhody:

\begin{itemize}
\item \textbf{Eliminácia globálneho výpadku (Single Point of Failure):} Ak dôjde k~poruche riadiacej jednotky v~jednej miestnosti, zvyšok múzea pokračuje v~prevádzke bez obmedzení.

\item \textbf{Zjednodušená údržba:} Pri servise v~jednej časti expozície nie je nutné odstavovať celý objekt.

\item \textbf{Opakovateľnosť:} Riešenie funguje ako univerzálny modul, ktorý je možné duplikovať do ľubovoľného počtu miestností.
\end{itemize}

\subsection{Hardvérová architektúra}
\label{hardverova-architektura}

Hardvérový návrh rozdeľuje systém na centrálnu logickú jednotku a~distribuované periférie. Na základe teoretickej analýzy v~kapitole~\ref{SucasnyStav} boli pre realizáciu prototypu zvolené konkrétne komponenty:

\textbf{Centrálna jednotka (Master):} Zvolený bol minipočítač Raspberry Pi 4. Táto voľba vychádza z~potreby súbežného behu MQTT brokera, webového servera a~hardvérovo akcelerovaného prehrávania videa, čo mikrokontroléry nezvládnu a~priemyselné PLC by predražili.

\textbf{Výkonové riadenie motorov (Driver):} Pre ovládanie mechanických častí (napr. dverí vitríny) bol na základe porovnania v~časti~\ref{aktuatory} vybraný modul s~čipom BTS7960.

\begin{itemize}
\item \textbf{Zdôvodnenie výberu:} Oproti bežnému modulu L298N, ktorý bol zvažovaný v~úvode, poskytuje BTS7960 prúdovú zaťažiteľnosť až 43\,A. To systému dodáva potrebnú robustnosť. Kľúčovým faktorom bol masívny chladič a~tepelná odolnosť, ktorá umožňuje inštaláciu aj do stiesnených priestorov múzejného nábytku bez rizika prehriatia.
\end{itemize}

\textbf{Distribuované uzly (Slaves):} Ako riadiace jednotky pre jednotlivé efekty boli zvolené čipy ESP32, ktoré zabezpečujú bezdrôtovú komunikáciu a~generovanie PWM signálov pre drivery.

\section{Návrh softvérovej logiky}
\label{navrh-softverovej-logiky}

Softvérové riešenie nie je postavené na pevne zakódovaných sekvenciách, ale využíva abstraktný model riadenia.

\subsection{Koncept stavového automatu (State Machine)}
\label{stavovy-automat}

Pre riadenie expozície bol zvolený model konečného stavového automatu (FSM). Systém sa v~každom okamihu nachádza v~jednom definovanom stave (napr. \uv{Intro}, \uv{Hra}, \uv{Idle}). Prechod medzi stavmi je riadený udalosťami (uplynutie času videa, stlačenie tlačidla). Tento prístup umožňuje nielen lineárne časovanie, ale aj zložité vetvenie deja.

\subsection{Návrh dátového modelu (JSON štruktúra)}
\label{datovy-model}

Konfigurácia správania expozície je uložená v~externých súboroch JSON, ktoré definujú kompletnú logiku stavového automatu pre danú scénu. Na rozdiel od jednoduchých lineárnych zoznamov, tento formát umožňuje definovať viacero stavov (States) a~prechody medzi nimi.

Koreňový objekt obsahuje identifikátor scény, počiatočný stav (initialState) a~objekt states, ktorý definuje jednotlivé fázy scenára. Každý stav môže obsahovať tri kľúčové polia:

\begin{itemize}
\item \texttt{onEnter}: Zoznam akcií, ktoré sa vykonajú okamžite pri vstupe do stavu (napr. zapnutie svetla).

\item \texttt{timeline}: Časová os pre akcie, ktoré sa majú vykonať s~oneskorením (definovaným parametrom \texttt{at}).

\item \texttt{transitions}: Podmienky pre prechod do iného stavu (napr. po uplynutí času \texttt{timeout}).
\end{itemize}

Ukážka reálnej štruktúry zo súboru \texttt{test.json}:

\begin{verbatim}
{
  "sceneId": "test_main",
  "initialState": "intro",
  "states": {
    "intro": {
      "description": "Začiatok - svetlo, audio, motor1",
      "onEnter": [
        {"action": "mqtt", "topic": "room1/light", "message": "ON"},
        {"action": "audio", "message": "PLAY:welcome.mp3:0.7"},
        {"action": "mqtt", "topic": "room1/motor1", "message": "ON:50:L"}
      ],
      "timeline": [
        {"at": 3.0, "action": "video", "message": "PLAY_VIDEO:ghost2.mp4"},
        {"at": 4.0, "action": "mqtt", "topic": "room1/motor1", 
         "message": "DIR:R"}
      ],
      "transitions": [
        {"type": "timeout", "delay": 6.0, "goto": "middle"}
      ]
    },
    "middle": {
      "description": "Stredná časť - vypne svetlo, zmení smer",
      "onEnter": [
        {"action": "mqtt", "topic": "room1/light", "message": "OFF"}
      ],
      "transitions": [
        {"type": "timeout", "delay": 2.0, "goto": "finale"}
      ]
    }
  }
}
\end{verbatim}

Tento model umožňuje flexibilné vetvenie deja, kde sa systém presúva medzi stavmi (intro $\rightarrow$ middle $\rightarrow$ finale) na základe definovaných prechodov.

\subsection{Návrh komunikačného rozhrania (MQTT Topics a~Payload)}
\label{mqtt-rozhranie}

Pre komunikáciu medzi centrálnou jednotkou a~perifériami bol zvolený jednoduchý textový protokol prenášaný cez MQTT. Namiesto komplexných JSON objektov v~tele správy (Payload) systém využíva reťazce s~parametrami oddelenými dvojbodkou (:). Tento prístup znižuje nároky na parsovanie v~mikrokontroléroch ESP32.

Adresácia zariadení využíva hierarchickú štruktúru tém v~tvare \texttt{miestnosť/zariadenie}.

Príklady použitých tém a~príkazov:

\begin{itemize}
\item \textbf{Ovládanie motorov:}
  \begin{itemize}
  \item Téma: \texttt{room1/motor1} (resp. \texttt{room1/motor2})
  \item Príkaz \texttt{ON}: Spustenie motora.
    \begin{itemize}
    \item Formát: \texttt{ON:Rýchlosť:Smer[:Trvanie]}
    \item Príklad: \texttt{ON:50:L} (Zapni na 50\,\% výkonu doľava)
    \item Príklad s~rampou: \texttt{ON:100:L:10000} (Rozbeh na 100\,\% po dobu 10 sekúnd)
    \end{itemize}
  \item Príkaz \texttt{DIR}: Zmena smeru za chodu.
    \begin{itemize}
    \item Príklad: \texttt{DIR:R} (Zmeň smer doprava)
    \end{itemize}
  \item Príkaz \texttt{OFF}: Okamžité zastavenie.
    \begin{itemize}
    \item Príklad: \texttt{OFF}
    \end{itemize}
  \end{itemize}

\item \textbf{Ovládanie osvetlenia:}
  \begin{itemize}
  \item Téma: \texttt{room1/light}
  \item Príklad: \texttt{ON} (Rozsvietiť)
  \item Príklad: \texttt{OFF} (Zhasnúť)
  \end{itemize}

\item \textbf{Multimédiá (Interné správy systému):}
  \begin{itemize}
  \item Akcie typu \texttt{audio} alebo \texttt{video} v~JSON scenári nie sú posielané priamo na MQTT, ale spracováva ich centrálna jednotka.
  \item Príklad audio: \texttt{PLAY:welcome.mp3:0.7} (Prehraj súbor s~hlasitosťou 70\,\%)
  \item Príklad video: \texttt{PLAY\_VIDEO:ghost2.mp4}
  \end{itemize}
\end{itemize}

Táto štruktúra príkazov umožňuje operátorovi alebo testovaciemu skriptu jednoducho ovládať zariadenia aj manuálne pomocou bežného MQTT klienta, keďže príkazy sú ľahko čitateľné a~zapisovateľné (human-readable).

\section{Implementácia centrálnej riadiacej jednotky (Backend)}
\label{implementacia-backend}

Implementačná časť popisuje realizáciu serverovej časti na platforme Raspberry Pi 4 s~operačným systémom Raspberry Pi OS.

\subsection{MQTT Broker a~služby}
\label{mqtt-broker}

Pre distribúciu správ bol nakonfigurovaný open-source broker Eclipse Mosquitto, optimalizovaný pre nízku latenciu. Broker prijíma spojenia na štandardnom porte 1883.

\subsection{Aplikačný backend (Python)}
\label{aplikacny-backend}

Jadrom systému je aplikácia v~jazyku Python, ktorá implementuje logiku stavového automatu. Využíva knižnicu \texttt{paho-mqtt} a~beží v~hlavnej slučke (event loop), ktorá vykonáva:

\begin{enumerate}
\item Kontrola času: Monitoruje trvanie aktuálneho stavu.
\item Exekúcia Timeline: Vykonáva naplánované akcie.
\item Spracovanie vstupov: Reaguje na správy zo senzorov.
\end{enumerate}

Súčasťou je modul Multimedia Handler, ktorý cez systémové nástroje (\texttt{omxplayer}, \texttt{vlc}) ovláda audio/video výstup a~synchronizuje ho s~MQTT príkazmi.

[SEM VLOŽIŤ: Výpis kódu -- ukážka z~\texttt{state\_machine.py} alebo \texttt{main.py}]

\subsection{Systémová stabilita (Watchdog)}
\label{watchdog}

Pre zabezpečenie bezobslužnej prevádzky bola implementovaná služba \texttt{museum-watchdog}. Tento nezávislý proces periodicky kontroluje beh aplikácie, dostupnosť brokera a~teplotu CPU. Pri detekcii chyby automaticky reštartuje služby.

\section{Implementácia koncových uzlov (Firmvér)}
\label{implementacia-firmware}

Pre bezdrôtové ovládanie periférií boli použité mikrokontroléry ESP32. Firmvér bol vyvinutý v~jazyku C++ s~využitím frameworku Arduino.

\subsection{Spracovanie správ a~JSON Parsing}
\label{json-parsing}

Prichádzajúce správy sú deserializované knižnicou ArduinoJson. Príklad príkazu pre motor:

\begin{verbatim}
{ "action": "ON", "speed": 50, "direction": "LEFT", "duration": 5000 }
\end{verbatim}

Firmvér na základe kľúčov v~JSON objekte nastavuje PWM signály na GPIO pinoch.

[SEM VLOŽIŤ: Výpis kódu -- parsovanie JSONu]

\subsection{Bezpečnostné mechanizmy (Fail-Safe)}
\label{fail-safe}

Firmvér obsahuje funkciu \uv{Fail-Safe}. Ak zariadenie stratí spojenie s~MQTT brokerom na viac ako 5 sekúnd, automaticky prejde do bezpečného stavu (zastavenie motorov), aby sa predišlo poškodeniu exponátov.

\section{Implementácia webového rozhrania}
\label{webove-rozhranie}

Keďže centrálna jednotka (Raspberry Pi) je v~expozícii často umiestnená v~technickom zázemí bez pripojeného monitora a~klávesnice (tzv. headless režim), bolo nutné vytvoriť nástroj na jej vzdialenú správu. Pre tento účel bola implementovaná ľahká webová aplikácia, ktorá slúži ako ovládací pult (Dashboard).

\subsection{Architektúra aplikácie}
\label{architektura-aplikacie}

Webové rozhranie je postavené na mikro-frameworku Flask (Python), ktorý beží priamo na centrálnej jednotke. Táto voľba zabezpečuje bezproblémovú integráciu s~backendom riadiaceho systému, keďže oba bežia v~rovnakom jazykovom prostredí.

\begin{itemize}
\item \textbf{Backend (Flask):} Spracováva HTTP požiadavky z~prehliadača a~komunikuje s~hlavným riadiacim procesom (State Machine) prostredníctvom zdieľaných systémových prostriedkov alebo lokálneho API.

\item \textbf{Frontend (HTML/CSS/JS):} Používateľské rozhranie je navrhnuté minimalisticky s~dôrazom na čitateľnosť na mobilných zariadeniach (tabletoch), ktoré personál múzea používa. Využíva asynchrónne volania (AJAX) na pravidelnú aktualizáciu stavu bez nutnosti obnovovania stránky.
\end{itemize}

\subsection{Funkcionalita prevádzkového Dashboardu}
\label{funkcionalita-dashboardu}

Implementovaný Dashboard poskytuje tri kľúčové funkcie nevyhnutné pre dennú prevádzku:

\begin{enumerate}
\item \textbf{Indikácia stavu v~reálnom čase:} Operátor vidí, či je systém pripravený, alebo či práve prebieha scéna. Aplikácia vizualizuje aj systémové logy, čo umožňuje rýchlu diagnostiku v~prípade problémov (napr. \uv{Chyba pripojenia k~MQTT brokeru}).

\item \textbf{Manuálny spúšťač scén (Scene Selector):} Aplikácia načíta zoznam dostupných konfiguračných súborov (JSON) z~diskovej pamäte a~ponúkne ich operátorovi vo forme tlačidiel. To umožňuje sprievodcom spustiť konkrétny efekt alebo výklad na vyžiadanie návštevníka, mimo automatického cyklu.

\item \textbf{Bezpečnostné zastavenie (STOP ALL):} V~rozhraní je implementované prioritné tlačidlo \uv{STOP}, ktoré odosiela signál na okamžité ukončenie všetkých bežiacich vlákien, vypnutie motorov a~zhasnutie svetiel. Táto funkcia je softvérovou obdobou bezpečnostného hríbu a~slúži na prevenciu škôd v~nepredvídaných situáciách.
\end{enumerate}

[SEM VLOŽIŤ OBRÁZOK: Screenshot z~webového rozhrania (Dashboardu) bežiaceho v~prehliadači, kde vidno tlačidlá scén a~tlačidlo STOP]

\subsection{Experimentálny generátor scén (SceneGen)}
\label{scenegen}

Nad rámec základných požiadaviek na prevádzku bol v~rámci práce vyvinutý aj prototyp grafického editora SceneGen (postavený na knižnici React.js). Tento nástroj slúži na vizuálnu tvorbu JSON súborov pomocou časovej osi (Timeline). Hoci nie je určený pre dennú obsluhu návštevníkmi, demonštruje možnosť budúceho rozšírenia systému o~používateľsky prívetivé konfiguračné rozhranie pre kurátorov výstav.

%
%\input{Zaver}% nutné
\begin{thebibliography}{99}
%
%  seznam literatury musi byt setriden podle abecedy
%

%
%  monografie
%
\bibitem{monografie}{BLATT, F.J.: {\it Modern Physics}. Ne w York: McGraw-Hill, 1992. 517 p. ISBN
0-07-005877-6.}

%
%  cast monografie
%
\bibitem{castmonografie}{MARÉCHAL, A, LOSTIS, P. and SIMON, J.: A precision interferometer with high light-gathering power. In VAN HEEL,
A.C.D.: {\it Advanced optical techniques}. Amsterdam: North-Holland Publishing,
1967, p. 435-446. ISBN 0-07-004857-5.}

\bibitem{castmonografie2}{PAVELEK, M. and LIŠKA, M.: A study of heat transfer from a horizontal vibrating
cylinder by means of holographic interferometry. In {\it Optical methods in
dynamics of fluids and solids}. Proccedins  of international symposium, Liblice,
1984, editor M. PLÍCHAL. Berlin: Springer, 1985, p. 43-49. ISBN 3-540-15247-4.}

%
%  clanek
%
\bibitem{clanek}{McNULTY, I.: The future of X-ray holography. {\it Nucl. Instr. and Meth. In Phys. Res. A}, 347, 1994, p.170-176. ISSN \ldots.}

%
%  clanek, vice nez 3 autori
%
\bibitem{clanekvicnez3autori}{RITUCCI, A., et al.: Damage and ablation of large bandgap dielectrics induced by a 46.9 nm laser beam.
{\it Opt. Lett.}, January 2006, vol. 31, no. 1, p. 68-71. ISSN \ldots.}

%
%  diplomova prace disertace
%
\bibitem{disertace}{ZLÁMAL, J.: {\it Simulace elektrostatických iontově optických systémů}. [Disertační práce.] Brno: VUT, FSI, 2003. 64 s.}

\end{thebibliography}
% nutné
\chapter{Seznam použitých zkratek a symbolů}
\symbolsize=3.5cm% sirka sloupecku pro symboly, je mozno zmensit pokud jsou kratke

\begin{symboly}
\item[$\Phi$] popis symbolu

\item[dlouhý symbol] popis symbolu

\item[ještě delší dlouhý symbol] popis symbolu
\end{symboly}
% nutné
\input{SeznamPriloh}% není povinné
\end{document}
